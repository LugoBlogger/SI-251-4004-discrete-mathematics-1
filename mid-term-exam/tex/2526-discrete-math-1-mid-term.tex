\documentclass[a4paper,12pt]{article}
\usepackage{fancyhdr}
\usepackage{graphicx}
\usepackage{amsmath}
\usepackage[nomarginpar, top=1cm, left=1cm, right=1.2cm, bottom=1cm]{geometry}

\usepackage[scaled]{helvet}
\usepackage[T1]{fontenc}
\renewcommand{\familydefault}{\sfdefault}
\usepackage{mathpazo}
\usepackage[bb=pazo]{mathalpha}
\usepackage{tabularx}

\pagenumbering{gobble}

\newcommand{\ITKheader}[3]{%
  \begin{center}
    \begin{tabular}{|m{0.25\textwidth}|>{\centering\arraybackslash}m{0.695\linewidth}| }
      \hline
      \includegraphics[width=0.25\textwidth]{./Logo_ITK_hstack.png}
      & \large\textbf{MID-TERM EXAM} \par
        ACADEMIC YEAR #1 - #2 SEMESTER \par
        #3 STUDY PROGRAMME \par 
        INSTITUT TEKNOLOGI KALIMANTAN \par 
        \\ \hline
    \end{tabular}
  \end{center}
}


\newcommand{\courseDetail}[6]{%
\begin{center}
  \bgroup
  \def\arraystretch{1.5}
  \begin{tabular}{|ll|ll|}
    \hline
    Course Name       &: #1
      & Duration  &: #4 \\ \hline
    Number of Credits &: #2             
      & Date      &: #5 \\ \hline
    Lecturer          &: #3
      & Exam type &: #6 \\ \hline
  \end{tabular}
  \egroup
\end{center}
}


\begin{document}

\ITKheader{2025/2026}{ODD}{INFORMATION SYSTEMS}

\courseDetail{Discrete Mathematics 1}%
  {3 credits}%
  {Henokh Lugo Hariyanto, M.Sc.}%
  {120 minutes}%
  {Wednesday / October 15th, 2025}%
  {Open an A4 cheat sheet}

\textbf{SOAL PAKET A}

\begin{enumerate}
  \item Diberikan tiga buah himpunan $A$, $B$, dan $C$. Gambarkan diagram Venn
  \begin{enumerate}
    \item $A \setminus (B \cup C)$;

    \item $\overline{A} \cap (B \cup C)$;
    
    \item $\overline{A} \cap (C \setminus B)$;
  \end{enumerate}

  \item Misalkan $p$, $q$, dan $r$ masing-masing merupakan proposisi
    sebagai berikut

    $p$: Kamu mendapatkan nilai A pada ujian akhir di kelas matematika diskrit \\
    $q$: Kamu mengerjakan semua latihan di buku matematika diskrit \\
    $r$: Kamu mendapatkan nilai A di kelas matematika diskrit

    Tuliskan proposisi berikut menggunakan $p$, $q$, dan $r$ dan operator
    logika (termasuk negasi)
    \begin{enumerate}
      \item Kamu mendapatkan nilai A pada ujian akhir di kelas matematika diskrit, 
        tetapi kamu tidak mengerjakan semua latihan di buku matematika diskrit.  
      \item Kamu mendapatkan nilai A pada ujian akhir di kelas matematika diskrit, 
        kamu mengerjakan semua latihan di buku matematika diskrit, dan 
        kamu mendapatkan nilai A di kelas matematika diskrit.
      \item Untuk mendapatkan nilai A di kelas matematika diskrit, kamu perlu
        mendapatkan nilai A pada ujian ujian akhir di kelas matematika diskrit.
      \item Kamu mendapatkan nilai A pada ujian akhir di kelas matematika diskrit, 
        namun kamu tidak mengerjakan semua latihan di buku matematika diskrit;
        akan tetapi, kamu mendapatkan nilai A di kelas matematika diskrit
      \item Mendapatkan nilai A pada ujian akhir di kelas matematika diskrit, 
        dan mengerjakan semua latihan di buku matematika diskrit adalah
        syarat cukup untuk mendapatkan nilai A di kelas matematika diskrit 
      \item Kamu mendapatkan nilai A di kelas matematika diskirt jika dan 
        hanya jika kamu mengerjakan semua latihan di buku matematika 
        diskrit atau kamu mendapatkan nilai A di ujian akhir di kelas 
        matematika diskrit.
    \end{enumerate}

  \vfill
  \newpage

  \item Melalui definisi jarak Jaccard kita dapat menghitung "jarak"
    antara dua buah himpunan. Definisi jarak Jaccard $d_J(A, B)$
    antara dua himpunan $A$ dan $B$ diberikan oleh:   
    $$ d_J(A, B) = 1 - J(A, B)$$   
    dengan $J(A, B)$ adalah kesamaan Jaccard yang didefinisikan sebagai 
    $$ J(A, B) = |A \cap B| / |A \cup B|$$
    Melalui definisi diatas, tentukan $d_j(A, B)$ dan $J(A, B)$ dari 
    dua himpunan berikut
    \begin{enumerate}
      \item $A = \{1, 3, 5\}$, $B = \{2, 4, 6\}$
      \item $A = \{1, 2, 3, 4\}$, $B = \{3, 4, 5, 6\}$
    \end{enumerate}

  \item Pertimbangkan suatu fungsi-fungsi yang merupakan fungsi-fungsi dari 
    himpunan semua \textit{youtuber} di Indonesia dan dipetakan ke himpunan-himpunan 
    di (a) sampai (d)
    \begin{enumerate}
      \item kode unik untuk kanal \textit{youtuber}
      \item pendapatan per bulan \textit{youtuber} dibulatkan ke jutaan.
      \item nama kanal \textit{youtuber}
      \item \textit{total views} seluruh video yang dimiliki oleh \textit{youtuber}
        dibulatkan ke ribuan.
    \end{enumerate}
    Dalam kondisi seperti apakah fungsi-fungsi tersebut dapat menjadi fungsi
    satu-satu (fungsi injektif)? Perhatikan bahwa sebenarnya ada empat 
    fungsi di dalam soal ini dan jabarkanlah setiap fungsi tersebut kapan dia 
    menjadi fungsi injektif.
\end{enumerate}

\vfill

\begin{center}
  \textbf{Selamat mengerjakan}

  \vspace{12pt}
  
  \textit{"Future success isn't solely determined by academic grades but 
    by the lasting knowledge and understanding one retains beyond 
    formal education"}
\end{center}

\newpage

\ITKheader{2025/2026}{ODD}{INFORMATION SYSTEMS}

\courseDetail{Discrete Mathematics 1}%
  {3 credits}%
  {Henokh Lugo Hariyanto, M.Sc.}%
  {120 minutes}%
  {Wednesday / October 15th, 2025}%
  {Open an A4 cheat sheet}

\textbf{SOAL PAKET B}

\begin{enumerate}
  \item Diberikan tiga buah himpunan $P$, $Q$, dan $R$. 
    Gambarkan diagram Venn
  \begin{enumerate}
    \item $P \cap (Q \cup R)$;

    \item $\overline{P} \cap (Q \cup R)$;
    
    \item $P \cup (Q \setminus R)$;
  \end{enumerate}

  \item Misalkan $a$, $b$, dan $c$ masing-masing merupakan proposisi
    sebagai berikut
  
    $a$: Seekor beruang telah terlihat di daerah ini. \\ 
    $b$: Pendakian di jalur ini aman. \\
    $c$: Ada buah mangga yang matang di sepanjang jalur pendakian.

    Tuliskan proposisi berikut menggunakan $a$, $b$, dan $c$ dan 
    operator logika (termasuk negasi)
    \begin{enumerate}
      \item Ada buah mangga yang matang sepanjang jalur pendakian, namun 
        seekor beruang tidak nampak di daerah ini.
      \item Seekor beruang tidak nampak di daerah ini dan pendakian 
        di jalur ini aman, namun ada buah mangga yang matang di sepanjang 
        jalur pendakian.
      \item Jika ada buah manga yang matang di sepanjang jalur pendakian, 
        pendakian di jalur ini aman jika dan hanya jika seekor beruang 
        tidak nampak di daerah ini.
      \item Tidaklah aman melakukan pendakian di jalur ini, namun 
        seekor beruang tidak nampak di daerah ini dan ada buah mangga 
        yang matang di sepanjang jalur pendakian.  
      \item Agar pendakian di jalur ini aman, adalah perlu namun 
        tidak cukup bahwa sebuah mangga tidak matang di sepanjang 
        jalur pendakian, dan seekor beruang tidak terlihat di daerah ini. 
      \item Pendakian tidak aman di jalur ini apabila seekor beruang 
        telah terlihat di daerah ini dan ada buah mangga yang matang 
        di sepanjang jalur pendakian.
    \end{enumerate}

  \vfill
  \newpage

  \item Melalui definisi jarak Jaccard kita dapat menghitung "jarak"
    antara dua buah himpunan. Definisi jarak Jaccard $d_J(P, Q)$
    antara dua himpunan $P$ dan $Q$ diberikan oleh:   
    $$ d_J(P, Q) = 1 - J(P, Q)$$   
    dengan $J(P, Q)$ adalah kesamaan Jaccard yang didefinisikan sebagai 
    $$ J(P, Q) = |P \cap Q| / |P \cup Q|$$
    Melalui definisi diatas, tentukan $d_j(P, Q)$ dan $J(P, Q)$ dari 
    dua himpunan berikut
    \begin{enumerate}
      \item $P = \{1, 2, 3, 4, 5, 6\}$, $Q = \{1, 2, 3, 4, 5, 6\}$
      \item $P = \{1\}$, $Q = \{1, 2, 3, 4, 5, 6\}$
    \end{enumerate}

  \item Pertimbangkan suatu fungsi-fungsi yang merupakan fungsi-fungsi 
    dari himpunan mahasiswa di suatu kelas matematika diskrit dan dipetakan 
    ke himpunan-himpunan di (a) sampai (d)
    \begin{enumerate}
      \item nama kota tempat mahasiswa berasal 
      \item nilai akhir yang didapatkan mahasiswa: A, AB, B, BC, C, dan D.
      \item nama depan mahasiswa
      \item merk kendaraan bermotor yang dibawa oleh mahasiswa.
    \end{enumerate}
    Dalam kondisi seperti apakah fungsi-fungsi tersebut dapat menjadi fungsi
    satu-satu (fungsi injektif)? Perhatikan bahwa sebenarnya ada empat 
    fungsi di dalam soal ini dan jabarkanlah setiap fungsi tersebut kapan dia 
    menjadi fungsi injektif.
\end{enumerate}

\vfill

\begin{center}
  \textbf{Selamat mengerjakan}

  \vspace{12pt}
  
  \textit{"Future success isn't solely determined by academic grades but 
    by the lasting knowledge and understanding one retains beyond 
    formal education"}
\end{center}

\end{document}